\documentclass[12pt,a4paper]{article}
\usepackage[utf8]{inputenc}
\usepackage{graphicx}
\usepackage[margin=1in]{geometry}
\usepackage{array}
\usepackage{enumitem}
\usepackage{titlesec}
\usepackage{amsmath}
\usepackage{amsfonts}
\usepackage{amssymb}
\usepackage{mathptmx} % Better Times New Roman implementation
\usepackage[T1]{fontenc} % Better font encoding

% Customizing section titles to match the document's style (simple and bold)
\titleformat{\section}
  {\normalfont\Large\bfseries}{\thesection}{1em}{}
\titleformat{\subsection}
  {\normalfont\large\bfseries}{\thesubsection}{1em}{}
\titleformat{\subsubsection}
  {\normalfont\normalsize\bfseries}{\thesubsubsection}{1em}{}

% Remove page number on the first page (cover page)
\pagenumbering{gobble}

\begin{document}

% --- Cover Page ---
\begin{titlepage}
    \centering
    \vspace*{0.3cm}
    
    \includegraphics[width=5.5cm]{coventry-logo.png} % Coventry University Logo
    
    \vspace{1cm}
    
    {\Huge\bfseries COVENTRY UNIVERSITY}
    
    {\Large UNITED KINGDOM}
    
    \vspace{1cm}
    
    {\Large School of Computing}
    
    \vspace{1.2cm}
    
    {\huge\bfseries Project Report on}
    
    \vspace{0.3cm}
    
    {\Huge\bfseries ``User Experience Designing Coursework''}
    
    \vspace{1.2cm}
    
    {\Large done by}
    
    \vspace{0.3cm}
    
    {\Large E M A S B EKANAYAKA – COBSCCOMP242-065}
    
    {\Large COVID – 16110762}
    
    \vspace{0.8cm}
    
    {\Large Under Supervision of}
    
    \vspace{0.3cm}
    
    {\Large\bfseries Mr. Hazeer (Lecturer)}
    
    \vfill % Pushes NIBM logo and text to the bottom
    
    \includegraphics[width=3.5cm]{nibm-logo.png} % NIBM Logo
    
    \vspace{0.3cm}
    
    {\Large School of Computing and Engineering}
    
    {\Large\bfseries NATIONAL INSTITUTE OF BUSINESS MANAGEMENT}
    
    {\Large SRI LANKA}
    
\end{titlepage}

\newpage
\pagenumbering{arabic} % Start page numbering from here

% --- Coursework Content ---

\section*{Coursework Assessment}
\renewcommand{\labelitemi}{--} % Change bullet point style for Task 1, 2, 4

\subsection*{Module Name: User Experience Designing}

\vspace{1cm}

\section*{Task 1}

\subsubsection*{What is the primary goal of usability in UI/UX}
The main aim of usability in UI/UX is to ensuring users can achieve their objectives easily and effectively. [1, 2] This involve identifying any problems in the design that might stop users from having a good experience. [3, 5] Ultimately, it’s about creating products that are not just functional but also pleasant for people to use.

\subsubsection*{Name two key factors in building a user-friendly interface}
Two important factors for a user-friendly interface is clarity and consistency. [6, 10] A clear layout helps users understand the interface quickly, and consistency in design elements makes the system predictable and easier to learn, which reduce confusion. [6, 7]

\subsubsection*{How does performing a usability study contribute in a design revamp}
A usability study is very helpful for a design revamp because it show what parts of the current design is not working well for users. [14, 15] This informations help designers make changes that are based on actual user needs and behaviors, leading to a more effective and satisfying new design. [11, 12] It ensures resources are invested into changes that truly matter.

\subsubsection*{Challenges in conducting a usability study}
One big challenge is finding the right participants who truly represent the target users for the product; if the sample is not right, the findings may be skewed. [16, 20] Another difficulty can be to ensure the test environment reflect how people will actually use the product in their daily lives, as artificial settings may not capture true behavior. [16, 17] Budget and time constraints also often pose significant hurdles.

\vspace{1cm}

\section*{Task 2}

\subsubsection*{How can data help in UI/UX}
Data provides valuable insights into how users interacts with a product, for example, what features they use most or where they encounter problems. [21, 25] This helps designers to understand user needs better, identify specific pain points, and make design decisions based on evidence rather than just assumptions, which ultimately lead to better products. [23, 24]

\subsubsection*{What is the difference between qualitative data and quantitative data}
Qualitative data gives us insights into the 'why' behind user actions, like their feelings, opinions, or motivations, and it is usually in non-numerical form. [26, 29] Quantitative data, on other hand, is about numbers and statistics, such as how many users clicked a specific button or how long they might have stayed on a particular page; it tells you 'how much' or 'how often'. [27, 28]

\subsubsection*{How does A/B testing improve a design}
A/B testing help improve a design by allowing designers to compare two different versions of an interface element (like a button or a layout) to see which one performs better with users in terms of achieving a specific goal. [31, 33] This process takes out much of the guesswork and lets design decisions be made based on actual user behavior and preferences, ensuring changes are genuinely beneficial. [32, 34]

\subsubsection*{List down a few common data collection methods and explain each in brief}
\begin{itemize}
    \item \textbf{Surveys:} These collect information from a large number of users through a set of structured questions. Surveys are good for gathering both quantitative data (e.g., ratings, frequencies) and some qualitative data (e.g., open-ended responses about experiences). [36, 37]
    \item \textbf{User Interviews:} These are one-on-one discussions with users to obtain in-depth qualitative insights about their attitudes, beliefs, motivations, and experiences regarding a product. They allow for follow-up questions and a deeper understanding of user perspectives. [38, 39]
    \item \textbf{Usability Testing:} This involves observing users as they try to complete predefined tasks with a product or prototype. Its primary goal is to identify usability problems, collect qualitative feedback, and measure task success rates. [37, 38]
\end{itemize}

\vspace{1cm}

\section*{Task 3}
\textit{This task is group work and has been omitted as per instructions.}

\vspace{1cm}

\section*{Task 4}

\subsubsection*{How does accessibility impact in user experience}
Accessibility makes sure that people with different abilities (including visual, auditory, motor, or cognitive impairments) can use a product effectively, and this usually improve the overall user experience for all users, not just those with disabilities. [41, 44] For instance, clear navigation and readable text that help users with disabilities also make the site easier for everyone else to use and understand. [42, 45]

\subsubsection*{Impact of pervasive computing in UI/UX}
Pervasive computing, where technology is embedded in many everyday objects and environments, mean that UI/UX design must create interfaces that are more adaptable, context-aware, and often less obtrusive. [46, 49] The interaction needs to be natural and often blend seamlessly into the user's environment, moving beyond traditional screen-based interfaces. [47, 50] This presents new challenges for designers to ensure usability across a multitude of interconnected devices.

\subsubsection*{What is UCD in UI/UX}
UCD, or User-Centered Design, is a design approach where the needs, wants, and limitations of the end user are the central focus at every stage of the design and development process. [51, 53] It involves users actively throughout development, through research and testing, to ensure the final product is highly usable, accessible, and meets their actual requirements. [52, 56]

\subsubsection*{How does UI/UX improve a business}
Good UI/UX can significantly improve a business by increase customer satisfaction and fostering loyalty. [57, 60] When users have a positive and effortless experience with a website or app, they are more likely to continue using that product or service, recommend it to others, and convert into paying customers, which can lead to higher sales and a stronger, more reputable brand image. [58, 59] It can also reduce development costs by identifying issues early.

\end{document}